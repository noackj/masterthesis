\documentclass[a4paper,12pt]{report}
\usepackage[utf8]{inputenc} % Kodierung
\usepackage[ngerman]{babel} % Sprache
\usepackage{geometry} % to change the page dimensions
\geometry{left=2.5cm, right=2cm, top=3cm, bottom=3cm} % or letterpaper (US) or a5paper or....
% \geometry{margin=2in} % for example, change the margins to 2 inches all round
% \geometry{landscape} % set up the page for landscape
%   read geometry.pdf for detailed page layout information
\usepackage{graphicx}
\usepackage{float}
\usepackage{fancyhdr}
\usepackage[bottom,hang]{footmisc}
\usepackage{tabularx}
\usepackage{setspace} % Paket für Zeilenabstand
\usepackage{acronym} % Paket für Abkürzungsverzeichnis

%\setlength{\textwidth}{14cm}
%\setlength{\textheight}{25.5cm}
%\setlength{\topmargin}{-2.0cm}
%\setlength{\oddsidemargin}{0cm}
%\setlength{\evensidemargin}{0cm}

\fancypagestyle{plain}{
\fancyhead{}
\renewcommand{\headrulewidth}{0.0pt}}

\pagestyle{fancy}
\renewcommand{\chaptermark}[1]{\markboth{#1}{}}
\fancyhf{}
\fancyhead[R]{}
\fancyhead[L]{\textbf{\nouppercase\leftmark}}
\fancyfoot[R]{\thepage}
\fancyfoot[L]{}
\renewcommand{\headrulewidth}{0.5pt}

% 1.5 facher Zeilenabstand
\onehalfspacing
% weniger Silbentrennung aber dafür mehr Wortzwischenräume
\sloppy

\begin{document}

%===================================================================== Titlepage
\begin{titlepage}
\centering
\vfill
{\bfseries\Huge Masterarbeit}\\[2cm]
{\bfseries\Large Modellierung der Qualitätsmanagementprozesse}\\[0.2cm]
{\bfseries\Large für die Marktüberwachung und Vigilanz in der}\\[0.2cm]
{\bfseries\Large Entwicklung von Medizinprodukten unter}\\[0.2cm]
{\bfseries\Large Berücksichtigung der MDR EU 2017/745}\\
\vfill
vorgelegt von
\vfill
{\large Jens Noack}\\
\vfill
in Kooperation mit der\\[1cm]
{\large W.O.M. WORLD OF MEDICINE GmbH}\\[1cm]
\begin{center}
\begin{minipage}[c]{0.3\textwidth}
   \includegraphics[width  = 3cm]{Images/wom_logo}
  \end{minipage}
\begin{minipage}[c]{0.2\textwidth}
   \includegraphics[width  = 3cm]{Images/akad_logo}
  \end{minipage}
\end{center}
\vfill
\begin{center}\parbox{0cm}{\begin{tabbing}
xxxxxxxxxx \= xxxxxxxx \kill
Hochschule:\quad\quad\quad\quad\quad\quad\quad\quad\quad \= AKAD Bildungsgesellschaft \\
Studiengang: \> Wirtschaftsingenieurwesen \\
\> Master of Engineering \\
Matrikelnummer: \> 2929271 \\
Erstgutachter: \> Dr. Andrea Herrmann\\
Betreuer Firma: \> Dr. Jan Bischof
\end{tabbing}}
\end{center}
\end{titlepage}

%===================================================================== Kurzfassung
\addcontentsline{toc}{chapter}{Kurzfassung} %sorgt für eintrag ins inhaltsverzeichnis
\chapter*{Kurzfassung} %  *-> erstellt unnummeriertes chapter

Kurzfassung Inhalt äöü ÄÖÜ \cite[S. 15]{Freund2014}

%===================================================================== Verzeichnisse
\tableofcontents %Inhaltsverzeichnis
\listoffigures %Abbildungsverzeichnis
\listoftables %Tabellenverzeichnis
\chapter*{Abkürzungsverzeichnis} %  *-> erstellt unnummeriertes chapter
\begin{acronym}[XXXXX] %Option in eckigen Klammern ist längste Abkürzung
 \acro{BPM}{Business Process Management}
 \acro{BPMN}{Business Process Model and Notation}
 \acro{PMS}{Post Market Surveillance}
\end{acronym}

%===================================================================== Einleitung
\chapter{Einleitung}\label{chap:Einleitung}

%===================================================================== Cahpter 2
\chapter{Grundlagen}\label{chap:Grundlagen}

\section{Analyse, Visualisierung und Modellierung von Geschäftsprozessen}\label{sec:BPM}
Dieses Kapitel stellt die Grundlagen für das Management von Geschäftsprozessen dar. Dies stellt einen essentiellen Anteil am Hauptziel dieser Arbeit, der die Anpassung der Geschäftsprozesse auf neue externe Anforderungen verfolgt, dar.
\subsection{Business Process Management}\label{subsec:BPManagement}
Unternehmen müssen sich in Zeiten von Globalisierung auf ihre Kernkompetenzen konzentrieren. Geschäftsprozesse nehmen dabei einen hohen Stellenwert ein, da sie den Kern des unternehmerischen Handelns beschreiben

\subsection{Business Process Mapping}\label{subsec:BPMapping}
\subsection{Business Process Modeling}\label{subsec:BPModeling}
\subsection{BPMN 2.0}\label{subsec:BPMN}

\section{Regulatorische Anforderungen für Medizingeräte}\label{sec:RegRequ}

\subsection{Gründe und Bedeutung der Regulierung für Medizinprodukte}\label{subsec:Gruende}
\subsection{Überblick über die wichtigsten Normen}\label{subsec:UeberblickNormen}
\subsection{Auswirkungen auf die Entwicklung und den Produktlebenszyklus}\label{subsec:AuswirkungenAufEntwicklung}
\subsubsection{Requirements Engineering}
\subsubsection{Product Lifecycle Management}

\section{Qualitätsprozesse nach Markteinführung medizinischer Geräte}\label{sec:PMProzesse}
\subsection{Vigilance System}\label{subsec:Vigilance}
\subsection{Post Market Surveillance}\label{subsec:PMS}
\subsection{Post Market Clinical Follow-Ups}\label{subsec:PMCF}
\subsection{Integration in die Entwicklungsprozesse}\label{subsec:IntegrationInDevProzesse}

%===================================================================== Chapter 3
\chapter{Aufnahme des Status Quo}\label{chap:AufnahmeStatusQuo}

%===================================================================== Cahpter 4
\chapter{Anpassung des Prozesses unter Berücksichtigung der Medizinprodukteverordnung (MDR) EU 2017/745}\label{chap:AnpassungAnMDR}

%===================================================================== Chapter 5
\chapter{Integration und Umsetzung des neu modellierten Prozesses}\label{chap:Integration}

%===================================================================== Zusammenfassung
\chapter{Zusammenfassung}\label{chap:Zusammenfassung}

%===================================================================== Diskussion
\chapter{Diskussion}\label{chap:Diskussion}

%===================================================================== Vorlagen
%\chapter{}\label{chap:}
%\section{}\label{sec:}
%\subsection{}\label{subsec:}
%\subsubsection{Requirements Engineering}  -- besitzt keine Nummerierung und taucht nicht im toc auf
%\cite[S. X]{<reference>}
%\ac{KDE} % K Desktop Environment (KDE)

%===================================================================== Literaturverz.
\bibliography{literatur}
\bibliographystyle{alphadin}
%\bibliographystyle{abbrv}
% Übersicht unter: https://de.wikibooks.org/wiki/LaTeX-W%C3%B6rterbuch:_bibliographystyle

%===================================================================== Anhang
\appendix
\chapter[Anhang]{}
\newpage
\section{Anhang 1}

%===================================================================== Eidessattl. Vers.
\chapter*{Eidesstattliche Versicherung} %  *-> erstellt unnummeriertes chapter
Ich versichere, dass ich vorliegende Arbeit selbstständig verfasst, keine anderen als die angegebenen Quellen und Hilfsmittel benutzt sowie alle wörtlich oder sinngemäß übernommenen Stellen in der Arbeit gekennzeichnet habe.
\\[2cm]
\noindent\rule{0.35\textwidth}{0.3pt}\rule{0.2\textwidth}{0pt}\rule{0.45\textwidth}{0.3pt}
\\Ort, Datum\rule{0.418\textwidth}{0pt}Unterschrift
\end{document}